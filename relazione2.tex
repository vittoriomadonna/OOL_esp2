\documentclass{article}

\title{Relazione 2}

\begin{document}

\maketitle



\section{Strumenti di misura e approccio sperimentale}

In questa esperienza sono state misurate le posizioni degli elementi cardinali di un sistema ottico, nella fattispecie di una lente biconvessa, tramite l'utilizzo della legge dei punti coniugati [vero?] e di misure dell'ingrandimento trasversale.

Per fare ciò è stato impiegato un banco ottico con [una riga imposta sul suo fianco] con tre cavalieri, su cui sono stati montati la lente da studiare, un oggetto di cui si vuole studiare l'immagine e uno schermo su cui quest'ultima veniva proiettata.
L'oggetto era costituito da un [vetrino trasparente su cui era stampato un quadrato] e veniva illuminato tramite una sorgente LED posta dietro di esso.

[E' possibile ricavare la posizione relativa degli oggetti per differenza tra le posizioni del bordo sinistro dei rispettivi cavalieri, introducendo 0.5cm di incertezza. La posizione dell'immagine veniva ricavata spostando lo schermo per proiettarla nitidamente su di esso e misurando dunque la posizione di quest'ultimo. Con questo apparato, è possibile misurare anche la dimensione dell'immagine sullo schermo tramite un righello e ricavare l'ingrandimento trasversale della lente rapportandola alle dimensioni dell'oggetto.]

[Qualche cenno teorico su come vogliamo effettivamente ricavare f1,f2,x1,x2 dalle relazioni lineari di cui vogliamo effettuare i fit, in particolare giustificando il fatto che ci servono p*, q* e Gy]

Dopo aver posizionato la lente in una posizione [a circa metà del banco ottico], sono stati posti gli altri cavalieri in maniera tale da allineare i centri della lente e dell'oggetto [ponendo il centro dell'oggetto sull'asse ottico della lente?] su una linea parallela all'asse ottico.

Ci si è poi assicurati che i piani della lente, dell'immagine e dello schermo fossero tutti allineati tra loro e perpendicolari al banco ottico.

Per fare ciò, si è utilizzata la tecnica di osservare le differenze di messa a fuoco[/nitidezza] in diverse parti dell'immagine proiettata e modificando la posizione e l'inclinazione degli elementi in modo da renderla quanto più uniforme possibile.

Dopo tutti questi accorgimenti sono stati misurati, al variare della posizione dell'oggetto $p^*$ rispetto all'asse ottico, la posizione dell'immagine $q^*$ e il suo ingrandimento trasversale $G_y$, ottenendo i valori riportati in tabella []
[Per compensare l'effetto della profondità di fuoco, sono stati presi i valori max e min dell'intervallo in cui l'immagine proiettata sullo schermo risultava a fuoco]

[Poi dobbiamo scrivere la parte di fit e tabelle, 
NB riportare tutte le formule e tutti i risultati intermedi, inclusi A e B del fit.]

[Le posizioni dei piani principali sono in accordo, quindi si considera che la lente può essere approssimata come fine e si constata che le due lunghezze focali sono compatibili tra loro, scegliendo quindi ...]


\section{conclusioni}



\end{document}
